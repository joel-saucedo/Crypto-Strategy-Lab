\documentclass{article}
\usepackage{amsmath, amssymb, amsfonts}
\usepackage{graphicx}
\usepackage{hyperref}

\title{Volatility Clustering Strategy: Mathematical Foundation}
\author{Crypto-Strategy-Lab}
\date{\today}

\begin{document}

\maketitle

\section{Abstract}

[Brief description of the trading edge and mathematical foundation]

\section{Mathematical Framework}

\subsection{Core Equation}

[Insert main equation here]

\subsection{Statistical Hypothesis}

\textbf{Null Hypothesis:} [Insert H0]

\textbf{Alternative Hypothesis:} [Insert H1]

\section{Edge Derivation}

[Detailed mathematical derivation of why this edge should exist]

\subsection{Market Microstructure Basis}

[Explanation of market mechanism that creates the edge]

\subsection{Statistical Properties}

[Expected distribution of returns, Sharpe ratio bounds, etc.]

\section{Implementation Details}

\subsection{Signal Generation}

[Step-by-step algorithm for generating trading signals]

\subsection{Parameter Optimization}

[Description of parameter space and optimization objective]

\section{Risk Management}

\subsection{Position Sizing}

[Integration with Kelly framework and risk limits]

\subsection{Stop Loss Logic}

[Mathematical basis for stop loss placement]

\section{Expected Performance}

\subsection{Theoretical Bounds}

[Theoretical Sharpe ratio and DSR bounds]

\subsection{Sensitivity Analysis}

[Parameter sensitivity and regime dependence]

\section{Unit Test Specifications}

\subsection{Synthetic Data Tests}

[Description of synthetic data that should yield positive DSR]

\subsection{Mathematical Consistency}

[Tests that verify implementation matches equations]

\section{References}

[Academic references and prior work]

\end{document}
