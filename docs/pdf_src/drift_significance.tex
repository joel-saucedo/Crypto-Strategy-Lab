\documentclass{article}
\usepackage{amsmath, amssymb, amsfonts}
\usepackage{graphicx}
\usepackage{hyperref}

\title{Cramér-Rao Drift-Significance Filter: Mathematical Foundation}
\author{Crypto-Strategy-Lab}
\date{\today}

\begin{document}

\maketitle

\section{Abstract}

The Cramér-Rao Drift-Significance Filter leverages the fundamental statistical principle that the variance of any unbiased estimator of a parameter cannot be smaller than the Cramér-Rao lower bound. Applied to drift detection in financial time series, this creates a rigorous statistical filter that only signals when drift estimates achieve sufficient statistical significance to overcome the inherent noise floor.

\section{Mathematical Framework}

\subsection{Core Equation}

The strategy is built on the Cramér-Rao lower bound for drift estimation:

\begin{equation}
\text{Var}(\hat{\mu}) \geq \frac{\sigma^2}{N}
\end{equation}

Where $\hat{\mu}$ is the sample mean (drift estimator), $\sigma^2$ is the true variance, and $N$ is the sample size.

The t-statistic for testing drift significance:

\begin{equation}
t = \frac{\hat{\mu}}{SE(\hat{\mu})} = \frac{\hat{\mu}}{\sigma/\sqrt{N}}
\end{equation}

\subsection{Statistical Hypothesis}

\textbf{Null Hypothesis:} $H_0: \mu = 0$ (no drift)

\textbf{Alternative Hypothesis:} $H_1: \mu \neq 0$ (significant drift exists)

\section{Edge Derivation}

The trading edge emerges from the fundamental asymmetry between signal and noise in financial markets. The Cramér-Rao bound provides the theoretical minimum variance for any drift estimator, establishing a noise floor below which no estimator can perform.

\subsection{Market Microstructure Basis}

In cryptocurrency markets, true economic drift (due to adoption, regulatory changes, or fundamental shifts) creates persistent directional pressure that compounds over time. However, market microstructure noise, algorithmic trading, and mean-reverting arbitrage create high-frequency fluctuations that obscure short-term drift signals.

The Cramér-Rao framework separates these regimes by requiring that drift estimates achieve statistical significance beyond what random noise could produce. Only sustained directional pressure can generate t-statistics exceeding the critical threshold consistently.

\subsection{Statistical Properties}

For normally distributed returns with true drift $\mu$ and volatility $\sigma$:

\begin{itemize}
\item Expected t-statistic: $E[t] = \frac{\mu}{\sigma/\sqrt{N}}$
\item Variance of t-statistic: $\text{Var}(t) = 1$ (under $H_0$)
\item Critical value for $\alpha = 0.05$: $t_{crit} \approx 2.0$ (for large N)
\end{itemize}

\section{Implementation Details}

\subsection{Signal Generation}

The algorithm operates in rolling windows of length $N$ (typically 90 days):

\begin{enumerate}
\item Calculate sample statistics: $\hat{\mu} = \frac{1}{N}\sum_{i=1}^{N} r_i$, $\hat{\sigma} = \sqrt{\frac{1}{N-1}\sum_{i=1}^{N} (r_i - \hat{\mu})^2}$
\item Compute t-statistic: $t = \frac{\hat{\mu}}{\hat{\sigma}/\sqrt{N}}$
\item Calculate p-value: $p = 2 \cdot P(T_{N-1} > |t|)$ where $T_{N-1}$ is Student's t-distribution
\item Apply significance filter: Signal only if $|t| > t_{min}$ (typically 2.0)
\item Generate directional signal: $\text{signal} = \text{sign}(\hat{\mu})$ if significant, 0 otherwise
\end{enumerate}

\subsection{Parameter Optimization}

Key parameters for optimization:
\begin{itemize}
\item Lookback window $N$: Controls trade-off between statistical power and responsiveness
\item Minimum t-statistic $t_{min}$: Sets significance threshold (higher = more selective)
\item Confidence level: Determines critical values for hypothesis testing
\end{itemize}

\section{Risk Management}

\subsection{Position Sizing}

Position size scales with signal strength:
\begin{equation}
w = w_{base} \cdot \min\left(\frac{|t|}{t_{max}}, 1\right)
\end{equation}

Where $w_{base}$ is the base position size and $t_{max}$ is the practical upper bound for t-statistics (typically 5.0).

\subsection{Stop Loss Logic}

Dynamic stop-loss placement based on statistical confidence:
\begin{equation}
\text{Stop Distance} = k \cdot \hat{\sigma} \cdot \sqrt{\Delta t}
\end{equation}

Where $k$ is inversely related to signal strength, ensuring tighter stops for weaker signals.

\section{Expected Performance}

\subsection{Theoretical Bounds}

For a strategy with true edge $\mu > 0$ and detection threshold $t_{min}$:

\begin{itemize}
\item Hit rate: $P(|t| > t_{min} | \mu \neq 0)$ increases with $|\mu|/\sigma$
\item Expected Sharpe ratio: Bounded by $\sqrt{N} \cdot |\mu|/\sigma$ when fully invested
\item DSR expectation: Should exceed 0.95 when $|\mu|/\sigma > 0.3$ annually
\end{itemize}

\subsection{Sensitivity Analysis}

Strategy performance depends critically on:
\begin{enumerate}
\item Signal-to-noise ratio in the underlying asset
\item Persistence of drift over the lookback window
\item Market regime (trending vs. mean-reverting periods)
\item Parameter calibration to asset characteristics
\end{enumerate}

\section{Unit Test Specifications}

\subsection{Synthetic Data Tests}

Tests verify positive DSR on synthetic data with known properties:
\begin{itemize}
\item Trending regimes with $\mu = 0.05\% $ daily, $\sigma = 2\%$
\item Mixed regimes alternating between drift and noise
\item Regime-switching data with varying persistence
\end{itemize}

\subsection{Mathematical Consistency}

Critical tests include:
\begin{itemize}
\item Verification that $\text{Var}(\hat{\mu}) = \sigma^2/N$ for normal data
\item Correct t-statistic calculation matching scipy.stats
\item Proper handling of degrees of freedom in critical value calculation
\item Signal strength proportional to statistical significance
\end{itemize}

\section{References}

\begin{itemize}
\item Cramér, H. (1946). Mathematical Methods of Statistics. Princeton University Press.
\item Rao, C.R. (1945). Information and accuracy attainable in the estimation of statistical parameters. Bulletin of the Calcutta Mathematical Society, 37, 81-91.
\item Lo, A.W. \& MacKinlay, A.C. (1988). Stock market prices do not follow random walks: Evidence from a simple specification test. Review of Financial Studies, 1(1), 41-66.
\end{itemize}

\end{document}
